\documentclass{article}
\usepackage[T1]{fontenc}
\usepackage[utf8]{inputenc}
\usepackage[polish]{babel}
\usepackage{amsmath}
\usepackage{hyperref}
\hypersetup{
    colorlinks=true,
    linkcolor=blue,
    filecolor=magenta,      
    urlcolor=cyan,
 %   pdftitle={Overleaf Example}, 
    pdfpagemode=FullScreen,
    }
%{Informatyka stosowana 2024, I st., semestr VI}


\author{
	{Kacper Jagodziński, 242403} \\
	{Jakub Szymczak, 242548}\\ 
{Prowadzący: prof. dr hab. inż. Adam Niewiadomski}
}

\title{Komputerowe systemy rozpoznawania 2023/2024\\Projekt 1. Klasyfikacja dokumentów tekstowych}
\begin{document}
\maketitle


\section{Cel projektu}
Celem projektu jest stworzenie aplikacji do klasyfikacji artykułów przy użyciu metody k-NN (k-najbliższych sąsiadów). Aplikacja skupia się na dwóch głównych modułach: ekstrakcji cech oraz klasyfikacji przy użyciu k-NN. Projekt opiera się na zbiorze artykułów pobranych z \href{https://archive.ics.uci.edu/dataset/137/reuters+21578+text+categorization+collection}{Link do zbioru danych}.


\section{Klasyfikacja nadzorowana metodą $k$-NN.  Ekstrakcja cech, wektory cech}

\subsection{Metoda k-NN}

Metoda k-NN (k-najbliższych sąsiadów) jest algorytmem klasyfikacji, który opiera się na koncepcji, że obiekty podobne powinny być klasyfikowane razem. Odbywa się poprzez obliczenie odległości wektora, który chcemy sklasyfikować między wektorami wszystkich artykułów znajdujących się w zbiorze uczącym, posegregowaniu obliczonych odległości rosnąco, a następnie rozpatrzeniu wyniku w oparciu o $k$ najbliższych sąsiadów i na końcu zwrócenie proponowanej klasyfikacji. Jeśli ktoś chce dowiedzieć się więcej o algorytmie $k$-NN polecam sprawdzić \cite{introduction-machine-learning}\\
\indent W projekcie przyjmuje się następujące parametry:

\begin{itemize}
    \item Wartość $k$ - (liczba najbliższych sąsiadów) liczba najlepszych odległości, które algorytm bierze pod uwagę podczas dokonywania predykcji.
    \item Proporcje podziału zbioru artykułów na zbiór uczący i testowy.
    \item Zbiór cech wyekstrahowanych z artykułu, na podstawie których dokonywana jest klasyfikacja.
    \item Metryka i miara podobieństwa używana w metodzie k-NN.
\end{itemize}

\subsection{Ekstrakcja cech}
Po bliższym przyjrzeniu się tekstom, zdecydowaliśmy się na wyekstrahowanie następujących cech:

\begin{enumerate}
    \item \label{licznosc.krainy} Zsumowana liczność występowania nazw krain geograficznych dla każdego z 6 państw badanych liczona jest osobno. (Cecha liczbowa, rzeczywista)
    \item \label{licznosc.obiektow} Zsumowana liczność występowania nazw charakterystycznych obiektów dla każdego z 6 państw badanych liczona jest osobno. (Cecha liczbowa, rzeczywista)
    \item \label{licznosc.miast} Zsumowana liczność występowania nazw miast dla każdego z 6 państw badanych liczona jest osobno. (Cecha liczbowa, rzeczywista)
    \item \label{licznosc.osob} Zsumowana liczność występowania osób (imiona, nazwiska) skojarzonych z danym krajem dla każdego z nich liczona jest osobno. (Cecha liczbowa, rzeczywista)
    \item \label{czy.instytucja} Czy występuję nazwa jakiejś instytucji dla danego państwa. (Lista wartości z przedziału \{0,1\}, Cecha logiczna)
    \item \label{ile.slowo} Ile razy wystąpiło słowo kluczowe (charakterystyczne) dla danego państwa. (Cecha liczbowa, Naturalna)
    \item \label{ile.kontynent} Ile razy wystąpiła najczęściej występująca nazwa kontynentu. (Cecha liczbowa, Naturalna)
    \item \label{czy.stolica} Czy występuje stolica któregoś z 6 państw badanych dla których dokonujemy klasyfikacji \textit{(west-germany, usa, france, uk, canada, japan)}. (Lista wartości z przedziału \{0,1\}, Cecha logiczna)
    \item Pierwsza nazwa stolicy występująca w tekście. (Cecha tekstowa)
    \item Najczęściej występująca nazwa państwa. (Cecha tekstowa)
    \item \label{ile.dlugosc} Długość tekstu, policzona ilość słów. (Cecha liczbowa, Naturalna)
    \item \label{czy.data} Występowanie daty skojarzonej z którymś z 6 badanych państw. (Lista wartości z przedziału \{0,1\}, Cecha logiczna)
\end{enumerate}

\subsection{Wzory cech}
\begin{enumerate}
    \item Wzór stosowany do obliczenia wartości cech: cechy \ref{licznosc.krainy}, cechy \ref{licznosc.obiektow}, cechy \ref{licznosc.miast} i cechy \ref{licznosc.osob} 
    \begin{equation}
        f(n, N, l, L) = 
        \sum_{n=1}^{N} \frac{l}{L} 
    \end{equation}
    gdzie: \textit{n} - numer elementu dla danego państwa,  \textit{N} - ilość wszystkich elementów dla danego państwa, \textit{l} - liczba wystąpień elementu w tekście, \textit{L} - długość tekstu

    \item Wzór potrzebny do obliczenia listy wartości z przedziału \{0,1\} dla cech: cechy \ref{czy.instytucja}, \ref{czy.stolica}, \ref{czy.data}
    \begin{equation} \label{wzor.logiczny} 
        f(x) =
        \begin{cases} 
            1, & x \in X, \\
            0, & x \not \in X.
        \end{cases}
    \end{equation}
    gdzie: \textit{X} - Słownik cechy.
    Wzoru \ref{wzor.logiczny} trzeba użyć tyle razy aby wypełnić wszystkie wartości wektora danej cechy.

    \item Wzór stosowany do obliczenia wartości cech: cechy \ref{ile.slowo}, cechy \ref{ile.kontynent} i cechy \ref{ile.dlugosc}   
    \begin{equation}
        f(N) = N
    \end{equation}
    gdzie: \textit{N} - ilość wystąpień wszystkich elementów. 

\end{enumerate}



\section{Miary jakości klasyfikacji} 
Miary jakości klasyfikacji (Accuracy, Precision,
Recall, F1). We wprowadzeniu zaprezentować minimum teorii potrzebnej do realizacji
zadania, tak by inżynier innej specjalności zrozumiał dalszy opis. Należy podać {\bf konkretne wzory miar użyte w tym eksperymencie oraz krótko
opisać ich znaczenie i zakresy przyjmowanych wartości. Należy podać przykładowe
wartości każdej miary. Nie przepisuj
teorii, ale podaj link/przypis i opisz jak rozumiesz jej zastosowanie w tym konkretnym
zadaniu}. \\
\indent Stosowane wzory, oznaczenia z objaśnieniami znaczenia symboli użytych w
doświadczeniu. Oznaczenia jednolite w obrębie całego sprawozdania.  Opis zawiera przypisy do bibliografii zgodnie z
Polską Normą, (zob. materiały BG PŁ).\\
\noindent {\bf Sekcja uzupełniona jako efekt zadania Tydzień 03 wg Harmonogramu Zajęć
na WIKAMP KSR.}


\section{Metryki i miary podobieństwa tekstów w klasyfikacji}
Wzory, znaczenia i opisy symboli zastosowanych metryk z
przykładami. Wzory, opisy i znaczenia miar
podobieństwa tekstów zastosowanych w obliczaniu metryk dla wektorów cech z
przykładami dla każdej miary \cite{niewiadomski08}.  Oznaczenia jednolite w obrębie całego sprawozdania.  {\bf Podaj metryki i miary
podobieństwa nie z literatury (te wystarczy zacytować linkiem), ale konkretne ich
postaci stosowane w zadaniu. Jakie zakresy wartości przyjmują te miary i
metryki, co oznaczają ich wartości? Podaj przykładowe wartości dla przykładowych wektorów cech}. \\ 
\noindent {\bf Sekcja uzupełniona jako efekt zadania Tydzień 04 wg Harmonogramu Zajęć
na WIKAMP KSR.}

\section{Wyniki klasyfikacji dla różnych parametrów wejściowych}
Wstępne wyniki miary Accuracy dla próbnych klasyfikacji na ograniczonym zbiorze tekstów (podać parametry i kryteria
wyboru wg punktów 3.-8. z opisu Projektu 1.). 
\noindent {\bf Sekcja uzupełniona jako efekt zadania Tydzień 05 wg Harmonogramu Zajęć
na WIKAMP KSR.}


\section{Dyskusja, wnioski, sprawozdanie końcowe}

Wyniki kolejnych eksperymentów wg punktów 2.-8. opisu projektu 1.  Każdorazowo
podane parametry, dla których przeprowadzana eksperyment. 
Wykresy (np. słupowe) i tabele wyników
obowiązkowe, dokładnie opisane w ,,captions'' (tytułach), konieczny opis osi i
jednostek wykresów oraz kolumn i wierszy tabel.\\ 

{**Ewentualne wyniki realizacji punktu 9. opisu Projektu 1., czyli ,,na ocenę 5.0'' i ich porównanie do wyników z
części obowiązkowej**.Dokładne interpretacje uzyskanych wyników w zależności od parametrów klasyfikacji
opisanych w punktach 3.-8 opisu Projektu 1. 
Szczególnie istotne są wnioski o charakterze uniwersalnym, istotne dla podobnych zadań. 
Omówić i wyjaśnić napotkane problemy (jeśli były). Każdy wniosek/problem powinien mieć poparcie
w przeprowadzonych eksperymentach (odwołania do konkretnych wyników: wykresów,
tabel). \\
\underline{Dla końcowej oceny jest to najważniejsza sekcja} sprawozdania, gdyż prezentuje poziom
zrozumienia rozwiązywanego problemu.\\

** Możliwości kontynuacji prac w obszarze systemów rozpoznawania, zwłaszcza w kontekście pracy inżynierskiej,
magisterskiej, naukowej, itp. **\\

\noindent {\bf Sekcja uzupełniona jako efekt zadań Tydzień 05 i Tydzień 06 wg Harmonogramu Zajęć
na WIKAMP KSR.}


\section{Braki w realizacji projektu 1.}
Wymienić wg opisu Projektu 1. wszystkie niezrealizowane obowiązkowe elementy projektu, ewentualnie
podać merytoryczne (ale nie czasowe) przyczyny tych braków. 


\begin{thebibliography}{0}
\bibitem{tadeusiewicz90} R. Tadeusiewicz: Rozpoznawanie obrazów, PWN, Warszawa, 1991.  
\bibitem{niewiadomski08} A. Niewiadomski, Methods for the Linguistic Summarization of Data: Applications of Fuzzy Sets and Their Extensions, Akademicka Oficyna Wydawnicza EXIT, Warszawa, 2008.
\bibitem{introduction-machine-learning} ,,Introduction to Machine Learning'' autorstwa Alpaydin, E. (2004).
\end{thebibliography}

Literatura zawiera wyłącznie źródła recenzowane i/lub o potwierdzonej wiarygodności,
możliwe do weryfikacji i cytowane w sprawozdaniu. 
\end{document}
