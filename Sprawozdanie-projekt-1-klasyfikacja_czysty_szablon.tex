\documentclass{article}
\usepackage[T1]{fontenc}
\usepackage[utf8]{inputenc}
\usepackage[polish]{babel}
%{Informatyka stosowana 2020, I st., semestr VI}


\author{
	{Autor Pierwszy, nr albumu 1} \\
	{Autor Drugi, nr albumu 2}\\ 
{Prowadzący: tytuł/stopień imię nazwisko}
}

\title{Komputerowe systemy rozpoznawania 2020/2021\\Projekt 1. Klasyfikacja dokumentów tekstowych}
\begin{document}
\maketitle

Opis projektu ma formę artykułu naukowego lub raportu z zadania
badawczego (doświadczalnego/obliczeniowego0 (wg indywidualnych potrzeb związanych np. z
pracą inżynierską/naukową/zawodową). Kolejne sekcje muszą być numerowane i
zatytułowane. Wzory są numerowane, tablice są numerowane i podpisane nad
tablicą, rysunki są numerowane i podpisane pod rysunkiem. Podpis rysunku i
tabeli musi być wyczerpujący (nie ogólnikowy), aby czytelnik nie musiał sięgać do tekstu, aby go
zrozumieć.\\
\indent {\bf Wybrane sekcje (rozdziały sprawozdania) są uzupełniane wg wymagań w
opisie Projektu 1. i Harmonogramie Zajęć na WIKAMP KSR jako efekty zadań w~poszczególnych tygodniach}. 

\section{Cel projektu}
Zwięzły (2-3 zdania) opis
problemu badawczego/obliczeniowego,  uwzględniający część badawczą i
implementacyjną. {\bf Nie przepisuj literatury ani teorii -- napisz krótko jak
rozumiesz to co masz wykonać: jakie działania, na jakim zbiorze danych (link lub
przypis), jaki jest spodziewany efekt}.\\
\indent Zamieszczony opis (własny, nie skopiowany) zawiera
przypisy do literatury (bibliografii) zamieszczonej na końcu raportu/sprawozdania
zgodnie z~Polską Normą cytowania bibliografii (zob. materiały BG PŁ pt. ,,Bibliografia
załącznikowa'').\\
\noindent {\bf Sekcja uzupełniona jako efekt zadania Tydzień 02 wg Harmonogramu Zajęć
na WIKAMP KSR.}


\section{Klasyfikacja nadzorowana metodą $k$-NN.  Ekstrakcja cech, wektory cech}
Krótki opis metody $k$-NN: zasada działania, wymagane parametry wejściowe, format
i~znaczenie wyników/rezultatów. Opis własny z przypisami do literatury -- minimum
teorii potrzebnej do zadania, tak by inżynier innej specjalności zrozumiał dalszy
opis \cite{tadeusiewicz90}. {\bf Nie przepisuj literatury ani teorii -- napisz krótko jak
rozumiesz to co masz wykonać w tym konkretnym przypadku}.\\

Wyekstrahowane cechy liczbowe i tekstowe dokumentów, min. 10 cech, w tym min. 2
o wartościach tekstowych, wszystkie
opisane słownie oraz wzorami, z objaśnieniem oznaczeń i przykładami użycia, do
tego precyzyjny opis możliwych wartości, które przyjmuje dana cecha (ułatwiający
czytelnikowi zrozumienie znaczenia w zadaniu klasyfikacji). Pamiętaj, że wybrane
cechy muszą reprezentować obiekt niezależnie od innych tekstów w tym samym lub w
innym zbiorze. Podaj postać wektora wartości cech po procesie ekstrakcji. Użyte oznaczenia są jednolite w całym
raporcie/sprawozdaniu. \\ 
\noindent {\bf Sekcja uzupełniona jako efekt zadania Tydzień 02 wg Harmonogramu Zajęć
na WIKAMP KSR.}

\section{Miary jakości klasyfikacji} 
Miary jakości klasyfikacji (Accuracy, Precision,
Recall, F1). We wprowadzeniu zaprezentować minimum teorii potrzebnej do realizacji
zadania, tak by inżynier innej specjalności zrozumiał dalszy opis. Należy podać {\bf konkretne wzory miar użyte w tym eksperymencie oraz krótko
opisać ich znaczenie i zakresy przyjmowanych wartości. Należy podać przykładowe
wartości każdej miary. Nie przepisuj
teorii, ale podaj link/przypis i opisz jak rozumiesz jej zastosowanie w tym konkretnym
zadaniu}. \\
\indent Stosowane wzory, oznaczenia z objaśnieniami znaczenia symboli użytych w
doświadczeniu. Oznaczenia jednolite w obrębie całego sprawozdania.  Opis zawiera przypisy do bibliografii zgodnie z
Polską Normą, (zob. materiały BG PŁ).\\
\noindent {\bf Sekcja uzupełniona jako efekt zadania Tydzień 03 wg Harmonogramu Zajęć
na WIKAMP KSR.}


\section{Metryki i miary podobieństwa tekstów w klasyfikacji}
Wzory, znaczenia i opisy symboli zastosowanych metryk z
przykładami. Wzory, opisy i znaczenia miar
podobieństwa tekstów zastosowanych w obliczaniu metryk dla wektorów cech z
przykładami dla każdej miary \cite{niewiadomski08}.  Oznaczenia jednolite w obrębie całego sprawozdania.  {\bf Podaj metryki i miary
podobieństwa nie z literatury (te wystarczy zacytować linkiem), ale konkretne ich
postaci stosowane w zadaniu. Jakie zakresy wartości przyjmują te miary i
metryki, co oznaczają ich wartości? Podaj przykładowe wartości dla przykładowych wektorów cech}. \\ 
\noindent {\bf Sekcja uzupełniona jako efekt zadania Tydzień 04 wg Harmonogramu Zajęć
na WIKAMP KSR.}

\section{Wyniki klasyfikacji dla różnych parametrów wejściowych}
Wstępne wyniki miary Accuracy dla próbnych klasyfikacji na ograniczonym zbiorze tekstów (podać parametry i kryteria
wyboru wg punktów 3.-8. z opisu Projektu 1.). 
\noindent {\bf Sekcja uzupełniona jako efekt zadania Tydzień 05 wg Harmonogramu Zajęć
na WIKAMP KSR.}


\section{Dyskusja, wnioski, sprawozdanie końcowe}

Wyniki kolejnych eksperymentów wg punktów 2.-8. opisu projektu 1.  Każdorazowo
podane parametry, dla których przeprowadzana eksperyment. 
Wykresy (np. słupowe) i tabele wyników
obowiązkowe, dokładnie opisane w ,,captions'' (tytułach), konieczny opis osi i
jednostek wykresów oraz kolumn i wierszy tabel.\\ 

{**Ewentualne wyniki realizacji punktu 9. opisu Projektu 1., czyli ,,na ocenę 5.0'' i ich porównanie do wyników z
części obowiązkowej**.Dokładne interpretacje uzyskanych wyników w zależności od parametrów klasyfikacji
opisanych w punktach 3.-8 opisu Projektu 1. 
Szczególnie istotne są wnioski o charakterze uniwersalnym, istotne dla podobnych zadań. 
Omówić i wyjaśnić napotkane problemy (jeśli były). Każdy wniosek/problem powinien mieć poparcie
w przeprowadzonych eksperymentach (odwołania do konkretnych wyników: wykresów,
tabel). \\
\underline{Dla końcowej oceny jest to najważniejsza sekcja} sprawozdania, gdyż prezentuje poziom
zrozumienia rozwiązywanego problemu.\\

** Możliwości kontynuacji prac w obszarze systemów rozpoznawania, zwłaszcza w kontekście pracy inżynierskiej,
magisterskiej, naukowej, itp. **\\

\noindent {\bf Sekcja uzupełniona jako efekt zadań Tydzień 05 i Tydzień 06 wg Harmonogramu Zajęć
na WIKAMP KSR.}


\section{Braki w realizacji projektu 1.}
Wymienić wg opisu Projektu 1. wszystkie niezrealizowane obowiązkowe elementy projektu, ewentualnie
podać merytoryczne (ale nie czasowe) przyczyny tych braków. 


\begin{thebibliography}{0}
\bibitem{tadeusiewicz90} R. Tadeusiewicz: Rozpoznawanie obrazów, PWN, Warszawa, 1991.  
\bibitem{niewiadomski08} A. Niewiadomski, Methods for the Linguistic Summarization of Data: Applications of Fuzzy Sets and Their Extensions, Akademicka Oficyna Wydawnicza EXIT, Warszawa, 2008.
\end{thebibliography}

Literatura zawiera wyłącznie źródła recenzowane i/lub o potwierdzonej wiarygodności,
możliwe do weryfikacji i cytowane w sprawozdaniu. 
\end{document}
